\documentclass[12pt,a4paper,russian]{moderncv} 
\usepackage[T2A]{fontenc} 
\usepackage[utf8]{inputenc} 


\moderncvstyle{classic}                        % style options are 'casual' (default), 'classic', 'oldstyle' and 'banking'
\moderncvcolor{green}                          % color options 'blue' (default), 'orange', 'green', 'red', 'purple', 'grey' and 'black'

% \usepackage[usenames]{xcolor}  %%% this does not work = thanks to idiots from modernCV package.
% 
\usepackage[scale=0.89]{geometry}
\setlength{\hintscolumnwidth}{2.5cm}           % if you want to change the width of the column with the dates


\firstname{Анна}
\familyname{Литвинцева}

\address{г. Санкт-Петербург}{\httplink[GitHub: N-31V]{github.com/N-31V}}
\mobile{+7(951)656-95-00}
\email{litvintseva\_ann@mail.ru}
\extrainfo{Telegram: n\_31v}
\quote{junior data scientist}
%\renewcommand\refname{Selected publications}
\nopagenumbers{} % uncomment to suppress automatic page numbering for CVs longer than one page

\renewcommand{\familydefault}{fca}
\renewcommand{\rmdefault}{cmss} % Шрифт с засечками
\renewcommand{\sfdefault}{cmss} % Шрифт без засечек
\renewcommand{\ttdefault}{cmtt} % Моноширинный шрифт

\begin{document}
\makecvtitle

\section{Образование}
\cventry{2016-2020}{ Университет ИТМО}{Санкт-Петербург}
{\newline Факультет: Cистем управления и робототехники 
\newline Образовательная программа: Интеллектуальные технологии в робототехнике
\newline Тема ВКР: Исследование архитектур нейронных сетей для классификации и локализации объектов на изображении (\httplink[github]{github.com/N-31V/VKR})}
{}{}

\cvline{Онлайн}{Нейронные сети (\httplink[сертификат stepik]{stepik.org/cert/177633})}
 
\cvline{курсы}{Нейронные сети и компьютерное зрение (\httplink[сертификат stepik]{stepik.org/cert/250485})}

\cventry{Учебный проект}
{Робот для обнаружения и транспортировки заданного объекта (\httplink[github]{github.com/N-31V/robotic-software})}{}
{\newline Распознавание объекта: сбор и разметка тренировочных данных, разработка и обучение свёрточной нейросети на python при помощи фреймворка TFLearn
\newline Навигация робота: установка альтернативной прошивки leJOS на ev3 и разработка алгоритма на java
\newline Интеграция: организация клиент-сервера между роботом и ПК (java-python)}
{}
{}

\cventry{2013-2015}{ Университет ИТМО}{Санкт-Петербург (неоконченное по собственному желанию)}
{\newline Факультет: Программной инженерии и компьютерной техники \newline Направление: Программная инженерия}{}{}

\section{Опыт работы}
\cventry{Январь 2017 -- Март 2020}{Старший лаборант}{Университет ИТМО}{\newline Обязанности: обеспечение технической поддержки аудиторного фонда}{}{}

\section{Навыки}
\cvline{Computer languages}{Python, C, Java, Scilab, MATLAB, asm, c\#, c++, SQL, javascript (в порядке убывания опыта)}
\cvline{ML tools}{NumPy, PyTorch, OpenCV, Pandas, Matplotlib, Seaborn}
\cvline{Other tools}{Linux, Bash, Git, \TeX}
\cvline{English}{Pre-Intermediate}

\section{Профессиональные интересы}
% \cventry{Date of birth}{June 20, 1982}{}{Russian Federation}{}{}
\cvline{Computer vision}{Самостоятельное изучение при помощи онлайн курсов, курс "основы технического зрения" в рамках учебной программы в университете, разработка нейронной сети (TFLearn) для проекта по поиску объекта роботом, исследование популярных архитектур (Pytorch) в рамках выпускной квалификационной работы.}
\cvline{Reinforcement learning}{В перспективе представляет собой наиболее интересное мне направление, но пока знакомство поверхностное.}
\end{document}